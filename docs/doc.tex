% This is a simple sample document.  For more complicated documents take a look in the exercise tab. Note that everything that comes after a % symbol is treated as comment and ignored when the code is compiled.

\documentclass{article} % \documentclass{} is the first command in any LaTeX code.  It is used to define what kind of document you are creating such as an article or a book, and begins the document preamble

\usepackage{amsmath} % \usepackage is a command that allows you to add functionality to your LaTeX code

\title{Music Genre Classification with Machine Learning} % Sets article title
\author{Erik Blenert} % Sets authors name
\date{\today} % Sets date for date compiled

% The preamble ends with the command \begin{document}
\begin{document} % All begin commands must be paired with an end command somewhere
    \maketitle % creates title using information in preamble (title, author, date)
    
    \section{Requirements} % creates a section
    
    
    The application will be trained with curated songs that are representative of the common music genres such as: 
    \begin{itemize}
        \item Electronic
        \item HipHop
        \item Rock
        \item Jazz
        \item etc
    \end{itemize}

    The application will be able to identify the genre of new a song or a list of songs in a folder given in CLI arguments. If the user wants, the application can write shown genres to the metadata of the MP3 files.
    
    If specified by the user, the application will also try to identify the energy level of the song on a scale of 1 to 5. A score of 1 means a chill, relaxed song. A score of 2-3 represents a good vibe.
A song that is danceable. A score of 4-5 means it's a banger. A very popular and high energy song.

    

\end{document} % This is the end of the document
